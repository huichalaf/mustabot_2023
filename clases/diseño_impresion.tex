\documentclass{article}
\usepackage[utf8]{inputenc}

\title{\textbf{Clase: Introducción a la Impresión 3D y Diseño con FreeCAD}}
\author{\textit{Team Mustabot 2022}}
\date{\today}

\begin{document}

\maketitle

\section*{Objetivo}
En esta clase, aprenderemos los fundamentos de la impresión 3D y el diseño utilizando la impresora 3D Ender 3 y el software FreeCAD en su modalidad de menú "Part". Al finalizar la clase, los estudiantes estarán familiarizados con los conceptos básicos de la impresión 3D y podrán utilizar FreeCAD para diseñar objetos en 3D y prepararlos para su impresión.

\section*{Duración}
1 hora

\section*{Agenda}
\subsection*{1. Introducción a la impresión 3D}
\subsubsection*{1.1 Definición y funcionamiento de una impresora 3D}
En esta sección, se explicará el proceso de impresión 3D, desde la creación del diseño en 3D hasta la fabricación de piezas utilizando una impresora 3D. Se detallarán los diferentes tipos de impresoras 3D y sus principios de funcionamiento, centrándose en el enfoque de deposición de material fundido (FDM). Recordaremos los conocimientos básicos previos sobre diseño e impresión para garantizar una base sólida para la clase.

\subsubsection*{1.2 Entender la finalidad del diseño y adaptarlo a las necesidades}
Antes de iniciar cualquier diseño en 3D, es fundamental comprender la finalidad y el uso previsto de la pieza que se va a imprimir. En esta sección, se discutirá la importancia de conocer los requisitos y limitaciones específicas del proyecto para adaptar el diseño en consecuencia. Abordaremos aspectos como el material a utilizar, la resistencia requerida, las tolerancias y otros factores que afectarán la impresión y el desempeño final de la pieza. Es esencial tener en cuenta que cada pieza tiene una finalidad única y su diseño e impresión deben ser específicos para cumplir con sus necesidades.

\subsection*{2. Preparación de archivos para la impresión 3D}
\subsubsection*{2.1 Exportación de diseños a Cura en formatos compatibles}
Una vez que el diseño en 3D está completo, es necesario exportarlo a un formato compatible con el software de preparación de impresión, como Cura. En esta sección, se explicará cómo exportar los diseños en formatos comunes, como STL y OBJ, para que puedan ser procesados en el software de slicing (Cura) y enviado a la impresora 3D. Repasaremos el proceso para que los estudiantes estén cómodos con la exportación de sus diseños.

\subsubsection*{2.2 Ajuste de configuraciones y generación de soportes}
Cada impresión 3D tiene requisitos específicos que deben configurarse adecuadamente en el software de slicing. En esta parte, abordaremos las configuraciones clave, como la velocidad de impresión, la densidad de relleno, la altura de capa y la temperatura del extrusor. También se explicará cómo generar estructuras de soporte para modelos que lo requieran, asegurando una impresión exitosa y evitando problemas de deformación. Recordaremos la importancia de optimizar tanto el material de impresión como el tiempo para lograr impresiones de alta calidad de manera eficiente.

\subsubsection*{2.3 Consideraciones de tiempo y costos en la impresión}
Además de los aspectos técnicos, es importante tener en cuenta el tiempo y los costos asociados con la impresión 3D. En esta sección, se analizarán factores como el tiempo de impresión estimado, el consumo de material y el costo por impresión. También se discutirán estrategias para optimizar la impresión y reducir los costos, como la impresión por lotes y el uso eficiente del material. Transmitiremos la importancia de planificar y considerar estos aspectos en el proceso de impresión 3D.

\section*{Nota}
Los niños tendrán un conocimiento base sobre diseño e impresión, por lo que esta clase tiene como objetivo dar un breve repaso y resolver dudas.

\end{document}
