\documentclass{article}
\usepackage[utf8]{inputenc}
\usepackage{listings}
\usepackage{xcolor}

\title{Documentación del Código}
\author{Team mustabot}
\date{\today}

\begin{document}

\maketitle

\section{Clase \texttt{MOTOR}}

Esta clase representa un motor en el robot. Proporciona métodos para controlar la velocidad del motor y obtener su nombre.

\begin{lstlisting}[language=Python, frame=single, backgroundcolor=\color{gray!10}]
class MOTOR:
    def __init__(self, robot, name):
        # Constructor
        ...

    def setVelocity(self, velocity):
        # Establece la velocidad del motor
        ...

    def getVelocity(self):
        # Devuelve la velocidad actual del motor
        ...

    def getName(self):
        # Devuelve el nombre del motor
        ...

    def stop(self):
        # Detiene el motor
        ...
\end{lstlisting}

\section{Clase \texttt{ENCODER}}

Esta clase representa un encoder en el robot. Proporciona métodos para leer el valor del encoder, establecer su posición inicial y obtener su nombre.

\begin{lstlisting}[language=Python, frame=single, backgroundcolor=\color{gray!10}]
class ENCODER:
    def __init__(self, robot, name):
        # Constructor
        ...

    def getValue(self):
        # Devuelve el valor actual del codificador
        ...

    def getName(self):
        # Devuelve el nombre del codificador
        ...

    def set_start_position(self):
        # Establece la posicion inicial del codificador
        ...
\end{lstlisting}

\section{Funciones \texttt{doblar\_derecha} y \texttt{doblar\_izquierda}}

Estas funciones calculan el tiempo y la potencia necesarios para que el robot gire hacia la derecha
o hacia la izquierda en una fracción de una vuelta completa. La fracción se determina 
mediante el valor de entrada a la función.

\begin{lstlisting}[language=Python, frame=single, backgroundcolor=\color{gray!10}]
def doblar_derecha(fraccion):
    # Calcula el tiempo y la potencia para girar a la 
    # derecha
    ...

def doblar_izquierda(fraccion):
    # Calcula el tiempo y la potencia para girar a la 
    # izquierda
    ...
\end{lstlisting}

\section{Función \texttt{normalize\_values}}

Esta función ajusta los valores de los sensores para mejorar el rendimiento del controlador PID cuando se enfrenta a colores como el verde.

\begin{lstlisting}[language=Python, frame=single, backgroundcolor=\color{gray!10}]
def normalize_values(values):
    # Ajusta los valores de los sensores
    ...
\end{lstlisting}

\section{Función \texttt{PID\_LINE}}

Esta función implementa un controlador PID para seguir una línea utilizando lecturas de sensores. Calcula el error, ajusta la velocidad de los motores en consecuencia y devuelve las velocidades actualizadas para el motor 1 y el motor 2.

\begin{lstlisting}[language=Python, frame=single, backgroundcolor=\color{gray!10}]
def PID_LINE(kp, kd, ki, previous_error, sensor_values
    PBASE, sum_error, motor1, motor2):
    # Implementa el controlador PID para seguir la linea
    ...
\end{lstlisting}

\section{Clases \texttt{IR}, \texttt{LASER} y \texttt{COLOR}}

Estas clases representan el sensor infrarrojo, el sensor láser y el sensor de color, respectivamente. Proporcionan métodos para leer los valores de los sensores y obtener los nombres de los sensores.

\begin{lstlisting}[language=Python, frame=single, backgroundcolor=\color{gray!10}]
class IR:
    def __init__(self, robot, name):
        # Constructor
        ...

    def getValue(self):
        # Devuelve el valor actual del sensor infrarrojo
        ...

    def getName(self):
        # Devuelve el nombre del sensor infrarrojo
        ...

class LASER:
    def __init__(self, robot, name):
        # Constructor
        ...

    def getValue(self):
        # Devuelve el valor actual del sensor laser
        ...

    def getName(self):
        # Devuelve el nombre del sensor laser
        ...

class COLOR:
    def __init__(self, robot, name):
        # Constructor
        ...

    def getValue(self):
        # Devuelve el valor actual del sensor de color
        ...

    def getName(self):
        # Devuelve el nombre del sensor de color
        ...

    def get_color(self):
        # Devuelve el color detectado por el sensor
        ...
\end{lstlisting}

\end{document}
