\documentclass[12pt]{article}
\usepackage{amsmath}
\usepackage{graphicx}
\usepackage{float}
\usepackage{listings}
\usepackage{xcolor}

\title{Movimiento}
\author{Team Mustabot 2022}
\date{\today}

\begin{document}
\maketitle

\section*{Funcion Mov}
La función mov(P1, P2), será nuestra gran aliada en lo que a movimiento del robot respecta, nos permite usando
PWM, cambiar la potencia que le entregamos a cada uno de los motores presentes en el robot, esto nos sirve para:
% Comentario: aca creamos una lista
\begin{itemize}
    \item Mover el robot hacia adelante
    \item Mover el robot hacia atrás
    \item Girar el robot hacia la derecha
    \item Girar el robot hacia la izquierda
\end{itemize}
Por otra parte, también existen otras funciones que nos permiten mover el robot usando movimientos predefinidos
tales como rotar(segundos) o frenar(segundos), estas funciones nos permiten mover el robot de manera más sencilla
pero también hacen uso de la función mov.

\section*{Funciones Derivadas}

\subsection*{Funcion mov\_forward}
Esta función nos permite mover el robot hacia adelante, para esto, se le entrega una sola potencia de 0 a 255 la cual se replica en ambos motores.

\subsection*{Funcion mov\_backward}
Esta función nos permite mover el robot hacia atrás, para esto, se le entrega una sola potencia de 0 a 255 la cual se replica en ambos motores.

\subsection*{Funcion mov\_right}
Esta función nos permite girar el robot hacia la derecha en su propio eje, para esto, se le entrega una sola potencia de 0 a 255 la cual se replica en ambos motores pero en sentido inverso.

\subsection*{Funcion mov\_left}
Esta función nos permite girar el robot hacia la izquierda en su propio eje, para esto, se le entrega una sola potencia de 0 a 255 la cual se replica en ambos motores pero en sentido inverso.

\subsection*{Funcion brake}
Esta funcion nos permite frenar el robot de una manera controlada, ingresamos como parametro el tiempo que queremos tarde en frenar por completo (en milisegundos)
% Puedes agregar más contenido entre las secciones si lo deseas

\end{document}
