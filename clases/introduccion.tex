\documentclass{article}
\usepackage[utf8]{inputenc}
\usepackage{hyperref}

\title{RoboCup y la Categoría Rescue Line Junior}
\author{Team Mustabot 2022}
\date{\today}

\begin{document}

\maketitle

\section{Introducción}
La RoboCup es una prestigiosa competencia internacional de Robótica, que reúne a equipos de todo el mundo para enfrentarse en diferentes categorías. Una de las categorías emocionantes y desafiantes es la denominada "Rescue Line Junior". En este artículo, exploraremos los detalles de esta categoría, sus objetivos y cómo los participantes deben superar diversos obstáculos para llevar a cabo tareas de rescate utilizando robots autónomos.

\section{La Competencia Rescue Line Junior}
La categoría Rescue Line Junior de RoboCup es una oportunidad para jóvenes entusiastas de la robótica para demostrar sus habilidades en el diseño y programación de robots inteligentes. En esta categoría, los equipos compiten utilizando robots capaces de moverse de forma autónoma siguiendo una línea trazada en el suelo.

\section{Objetivo de la Competencia}
El objetivo principal de la categoría Rescue Line Junior es simular una misión de rescate en un entorno peligroso y desafiante. Los robots deben ser capaces de seguir una ruta predefinida, enfrentar obstáculos y recoger objetos dispersos en el área para llevarlos a un lugar seguro.

\section{Desarrollo de la Competencia}
La competencia consta de dos partes principales:

\subsection{Parte 1: Seguir la Línea y Superar Obstáculos}
En esta etapa inicial, los robots deben demostrar su capacidad para seguir una línea en el suelo mientras enfrentan diversos obstáculos. Estos obstáculos pueden ser intersecciones con diferentes colores, rampas para subir y bajar, y obstáculos estilo "Speed Bounce", que son similares a los "monstruos de toro". Los robots deben tomar decisiones en tiempo real para cumplir con las instrucciones correspondientes a cada obstáculo.

\subsection{Parte 2: Rescate de Pelotas en un Área Insegura}
La segunda parte de la competencia se enfoca en tareas de rescate. Los robots deben recoger pelotas dispersas en un área considerada peligrosa y llevarlas a una zona segura. Existen dos niveles para esta tarea. En el primer nivel, el robot puede empujar la pelota hacia el área segura, mientras que en el segundo nivel, debe recogerla y depositarla en la zona segura. Se otorgan más puntos si el robot recoge la pelota correctamente en lugar de simplemente arrastrarla.

\section{Reglas y Puntuación}
La competencia Rescue Line Junior cuenta con reglas específicas y puntos asignados para cada tarea. Los robots tienen un tiempo limitado para completar todas las tareas y alcanzar los checkpoints. Si un robot comete un error, puede regresar al último checkpoint para intentarlo nuevamente.

\section{Aprendizaje y Participación}
La RoboCup Rescue Line Junior es una excelente oportunidad para que jóvenes estudiantes de robótica desarrollen sus habilidades en ingeniería, programación y trabajo en equipo. Los desafíos presentados en la competencia fomentan el pensamiento creativo y la resolución de problemas, lo que lleva a una experiencia de aprendizaje enriquecedora.

\section{Conclusión}
La competencia RoboCup y la categoría Rescue Line Junior brindan una plataforma emocionante para que jóvenes entusiastas de la robótica demuestren su talento y creatividad. Mediante la construcción y programación de robots autónomos, los participantes tienen la oportunidad de enfrentarse a desafíos del mundo real, como misiones de rescate, y aplicar sus conocimientos en un entorno competitivo y educativo.

\href{https://www.youtube.com/watch?v=GoGdpmvcXAc}{Video de ejemplo Rescue Line}

\end{document}
